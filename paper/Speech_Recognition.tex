\documentclass{article}
\usepackage[english]{babel}
\usepackage{amsmath}
% use geometry to set size to A4 and set margins around the document
\usepackage[a4paper, total={6in, 8in}]{geometry}

\begin{document}

\title{Microsoft Azure Speech Recognition}
\author{Robin Vonk - 500775219 \\
		Michel Rummens - 500778934 \\}
\date{15/12/2019}

% Distance between the two text columns
\setlength{\columnsep}{30px}

% Set text in two columns
\twocolumn[
% Disable two columns for title page and abstract
\begin{@twocolumnfalse}
\begin{center}
    \maketitle
    \vspace*{-0.8cm}
    
    \rule{0.9\textwidth}{0.1mm} 
    
    \begin{abstract}
        \normalsize 
        Abstract
        TODO Add abstract here
        \vspace*{0.3cm}
    \end{abstract}
    
    \rule{0.9\textwidth}{0.1mm} 

\end{center}

\end{@twocolumnfalse}
]

\tableofcontents
\section{Introduction}
Introduction, which contains a description of your application and your null \\
This paper will discuss the correctness of the Microsoft Azure Speech Recognition algorithm. This paper will not discuss the working nor any technical aspect of the algorithm. This paper will use a test set of audio fragments, run these tests through the algorithm and use the output for calculations.



\subsection{Main question}
TODO: Formulate a main question

\subsection{Hypothesis}
The speech to text algorithm of Microsoft is one of the best tested algorithm in this field\cite{Veton}. For this reason it is expected that this algorithm will do really well. It's results should little to not differ from original transcript of the audio files in any situation.

\subsection{About Microsoft Speech to Text}
Description of the application including the technique used by the underlying framework.

\section{Experiment}
Experimental set-up \\
Experimental setup, which describes how you conduct your experiment

\subsection{Resources used}
Microsoft Azure Speech Recognition

\subsection{Method}

\section{Results}
For the evaluation of you application you have to use at least the following metrics: \\
• Real Time Factor (RTF), Word Error Rate (WER), and Word Correct Rate (WCR) \\
• The micro and macro averages of the Recall and Precision and the F-score \\

Results, which contains the data you gathered during the experiment and the
figures and statistics you used for the interpretation of these data \\
\\
Speech recognition applications are usually evaluated by:

\begin{itemize}
    \item real time factor (RTF), the time needed to get result, i.e. the time of output minus the time of input. RTF depends on:
    \begin{itemize}
        \item used hardware (NB networks)
        \item length of the input sentence
        \item used vocabulary
    \end{itemize}
    \item word error rate (WER), the edit distance between two strings is the minimum number (or weighted sum) of insertions, deletions and substitutions required to transform one string into the other. For measuring WER you need:
    \begin{itemize}
        \item S = Substitutions: word is replaced
        \item D = Deletions: word is missed out
        \item I = Insertions: word is added
        \item C = Corrects: word matches
        \item N = Number of words in reference sentence
    \end{itemize}
\end{itemize}


\subsection{Data}
\subsection{Calculations}
    
    Word Error Rate: 
    \begin{center}
    $ WER = \frac{S + D + I}{N}$ \\
    \end{center} 
    
    Word Recognition Rate: 
    \begin{center}
    $ WRR = 1 - WER = \frac{C - I}{N}$ \\
    $ NBC =N - (S + D)$ \\
    \end{center} 
    
    Word Correct Rate: 
    \begin{center}
    $ WCR = \frac{C}{N}$ \\
    \end{center} 
    
    Precision: 
    \begin{center}
    $ Precision = \frac{relevant\ documents \cap retrieved\ documents}{retrieved\ documents} = \frac{|R_i \cap A_i|}{|A_i|} $ \\
    \end{center} 
    
    Recall : 
    \begin{center}
    $ Recall = \frac{relevant\ documents \cap retrieved\ documents}{relevant\ documents} = \frac{|R_i \cap A_i|}{|R_i|} $ \\
    \end{center} 
    
    F-score:
    \begin{center}
    $ Recall = \frac{2 \times \pi_i \times \rho_i}{\pi + \rho_i} $\\
    \end{center} 

\subsection{Graphs}

\section{Discussion}
The test set that was used was made by fellow students. This has the consequence that the quality of the tests varies within the test set. For example, one of the test sentences ends, according to the provided transcript, with 'in 619 AD'. However the algorithm recognized this as 'in 6 1980'. When we listened to this audio fragment ourselves, we also thought the fragment said 'in 1980'. This means that the quality of this paper may vary, because the used test set also varies in quality.

\section{Conclusion}
Conclusion, which concludes your paper and states nothing more than already
written

\bibliographystyle{ieeetr} 
\bibliography{References} 

\end{document}
